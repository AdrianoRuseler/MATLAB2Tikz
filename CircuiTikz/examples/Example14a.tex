% TikZ-Timing v0.7 2009/12/09 Example 8 & 9:
% Author: Martin Scharrer
%
% This is an example for the 'tikz-timing' package.
% It redraws the timing diagram taken from
% http://commons.wikimedia.org/wiki/File:SR_FF_timing_diagram.png
% http://en.wikipedia.org/wiki/File:SPI_timing_diagram.svg
%
\documentclass[border=3mm]{standalone}
\usepackage{tikz-timing}[2009/12/09]
% Use tikz-timing library 'counters' to define a counter character.
% This character draws a 'D{<counter value>}' and increases the counter
% value by one. A reset character which resets the counter value 
% (by default to 1) is also defined.
\usetikztiminglibrary[new={char=Q,reset char=R}]{counters}
\usepackage[active,tightpage]{preview}
\PreviewEnvironment{tikzpicture}
\setlength{\PreviewBorder}{5mm}
\pagestyle{empty}

\begin{document}
	%
	% Defining foreground (fg) and background (bg) colors
	\definecolor{bgblue}{rgb}{0.41961,0.80784,0.80784}%
	\definecolor{bgred}{rgb}{1,0.61569,0.61569}%
	\definecolor{fgblue}{rgb}{0,0,0.6}%
	\definecolor{fgred}{rgb}{0.6,0,0}%
	%
	\begin{tikztimingtable}[
		timing/slope=0,         % no slope
		timing/coldist=2pt,     % column distance
		xscale=2.05,yscale=1.1, % scale diagrams
		semithick               % set line width
		]
		 clock     & 7{C}                              \\
		S                     & [fgblue] .75L h 2.25L H LLl       \\
		R                     & [fgblue]  1.8L .8H 2.2L 1.4H 0.8L \\
		Q                     &          L .8H 1.7L 1.5H LL       \\
		$\overline{\mbox{Q}}$ &          H .8L 1.7H 1.5L HH       \\
		Q                     & [fgred]  HLHHHLL                  \\
		$\overline{\mbox{Q}}$ & [fgred]  LHLLLHH                  \\
	\end{tikztimingtable}%
	%
\end{document}